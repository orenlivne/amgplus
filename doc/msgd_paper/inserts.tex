%%%%%%%%%%%%%%%%%%%%%%%%%%%%%%%%%%%%%%%%%%%%%%%%%%%%%%%%%%%%%%%%%%%%%%%%%%%%%%%%%%%%%%
% Achi's revisions, organized in snippets. Eventually to be integrated
% into the the main multiscale optimization paper.
%%%%%%%%%%%%%%%%%%%%%%%%%%%%%%%%%%%%%%%%%%%%%%%%%%%%%%%%%%%%%%%%%%%%%%%%%%%%%%%%%%%%%%

\documentclass{article} % For LaTeX2e
%\documentstyle[nips14submit_09,times,art10]{article} % For LaTeX 2.09

%--------------------------------------------------------
% Packages.\dfrac{•}{•}
%--------------------------------------------------------
%\usepackage{nips15submit_e,times}
\usepackage[toc,page]{appendix}
\usepackage{hyperref}
\usepackage{url}
\usepackage{graphicx}
\usepackage{amsmath}
\usepackage{amsthm}
\usepackage{amssymb}
\usepackage{subcaption} 
\usepackage{mathtools}
%\usepackage{xcolor}
%\usepackage{soul}
\usepackage{graphicx}
\usepackage[disable]{todonotes}
\usepackage{changebar}
%\graphicspath{ {figs/} }
\newtheorem{theorem}{Theorem}
\include{aliases}
\newcommand{\ins}{INSERT}

% Number equations A.1, A.2, etc.
\counterwithin*{equation}{section}
\renewcommand{\theequation}{\thesection.\arabic{equation}}
%--------------------------------------------------------
% Title.
%--------------------------------------------------------
\title{Multi-level Stochastic Optimization and Neural Networks - Snippets}
% \author{
% Achi Brandt\\
% Oren E. Livne \\
% }
\date{\today}

%%%%%%%%%%%%%%%%%%%%%%%%%%%%%%%%%%%%%%%%%%%%%%%%%%%%%%%%%%%%%%%%%%%%%%%%%%%%%%%%%%%%%%%
% Paper starts here.
%%%%%%%%%%%%%%%%%%%%%%%%%%%%%%%%%%%%%%%%%%%%%%%%%%%%%%%%%%%%%%%%%%%%%%%%%%%%%%%%%%%%%%%
\begin{document}

% \maketitle
% \tableofcontents
% \todototoc
% \listoftodos

%\begin{abstract}
%\end{abstract}

%%%%%%%%%%%%%%%%%%%%%%%%%%%%%%%%%%%%%%%%%%%%%%%%%%%%%%%%%%%%%%%%%%%%%%%%%%%%%%%%%%%%%%%

\newpage
\section{Appendix X: Dependence of Gradient Combination Statistics on Input Activation Statistics}
\label{sec:x}

{\bf Notation.} In a given layer, at each output point $p$ and for each sample $m$, denote
\begin{itemize}
    \item $a^m_i(p)$ = the $i${\it th} post-ReLU {\it input} activation.
    \item $\gamma^m_j(p)$ = the loss derivative with respect to the $j${\it th} pre-ReLU
    {\it output} activation.
    \item $g^m_{ij} = \sum_p a^m_i(p) \gamma^m_j(p)$ = gradient of the weight $w_{ij}$.
    \item $\overline{[\cdot]^m}$ = averaging the quantity $[\cdot]^m$ over all samples $m$.
\end{itemize}

{\bf Combinations.} For any normalized vector $\alpha = (\alpha_1, \alpha_2, ...)$, $\sum_i \alpha_i^2 = 1$, we define a normalized combination of the input activations $A^m_{\alpha} = \sum_i \alpha_i a^m_i(p)$ and a corresponding gradient combination
$$ G^m_{\alpha j} = \sum_i \alpha_i \alpha_i g^m_{ij} = \sum_i \alpha_i \sum_p a^m_i(p) \gamma^m_j(p)\,.$$

{\bf Assumption 1:} during training gradient averages should tend to zero, so we assume $|\overline{\gamma^m_j(p)}| \ll \overline{|\gamma^m_j(p)|}$.

{\bf Averages.} $\overline{G^m_{\alpha j}} = \sum_p \overline{A^m_{\alpha}(p) \gamma^m_j(p)}\,.$
Since $\gamma^m_j(p)$, and even its sign, may vary arbitrarily, $\overline{G^m_{\alpha j}}$ cannot be estimated from input activation statistics. In contrast, the relative sizes of $\overline{(G^m_{\alpha j})^2}$ and variances of $G^m_{\alpha j}$ can generally be estimated from the input activation statistics, as follows.

{\bf Variance.}
\begin{equation}
    \begin{split}
    \overline{\left(G^m_{\alpha j} - \overline{G^m_{\alpha j}}\right)^2} \approx \overline{\left(G^m_{\alpha j}\right)^2} \\
     = & \sum_{p, q} \overline{A^m_{\alpha}(p) \gamma^m_j(p) A^m_{\alpha}(q) \gamma^m_j(q)} \\
     \leq & \frac14 \sum_{p, q} \overline{\left(A^m_{\alpha}(p)^2 + A^m_{\alpha}(q)^2\right)(\left(\gamma^m_j(p)^2 + \gamma^m_j(q)^2\right)}\footnote{Follows from the arithmetic-geometric mean inequality and Lagrange's identity: $abcd \leq ((ac + bd)/2)^2 \leq \frac14 ((ac + bd)^2 + (a d - b c)^2) = \frac14 (a^2 + b^2)(c^2 + d^2).$}\\
     = & \sum_p \overline{A^m_{\alpha}(p)^2 \cdot \frac12 \sum_q (\left(\gamma^m_j(p)^2 + \gamma^m_j(q)^2\right)}\footnote{I need further explanation on why this inequality and the next one hold.}\\
     \approx & N_{\text{out}} \sum_p \overline{A^m_{\alpha}(p)^2} \sum_p \overline{\gamma^m_j(p)^2}\,,
    \end{split}
    \label{g_var}
\end{equation}
where $\approx$ denotes likely comparable sizes. Indeed, dividing both sides by $N_{\text{out}} \sum_p \overline{\gamma^m_j(p)^2}$, they both express some weighted averages of $\sum_p A^m_{\alpha}(p)^2$, only with different non-negative weights.

{\bf Conclusion I.} A gradient combination can generally have large variance (compared with the variance of some other normalized combinations in the same output sheaf) only if the corresponding combination of input activations has a large size (compared to the size of some other activation combinations).

{\bf Conclusion II.} A small error in an interpolation between input activations implies a small error in the corresponding parameter interpolation in every sheaf. To see this, apply the above inequality to the combination expressing the interpolation {\it error}, which is the difference between the interpolant, which is a linear combination of coarse variables, and the target fine-level variable, where each coarse variable is itself a linear combination (the operator $Q$) of fine-level variables.

{\bf Fully-connected layer.} In the special case of a fully connected layer, where there is no out-point ($p$) dependence , the relations are simpler (approximate equality, not {\it in}equality) and extend to {\it co}variances (not just variances) of combinations, as follows.

{\bf Covariance.}
\begin{equation}
    \begin{split}
    \overline{\left(G^m_{\alpha j} - \overline{G^m_{\alpha j}}\right) \left(G^m_{\beta j} - \overline{G^m_{\beta j}}\right)} & \\
     & \approx \overline{G^m_{\alpha j} G^m_{\beta j}} \\
     & = \overline{A^m_{\alpha} A^m_{\beta} \left(\gamma^m_j\right)^2} \\
     & \approx \overline{A^m_{\alpha} A^m_{\beta}} \overline{\left(\gamma^m_j\right)^2}\,,
    \end{split}
    \label{g_cov}
\end{equation}
because, dividing both sides by $\overline{\left(\gamma^m_j\right)^2}$, both express some weighted average of $A^m_{\alpha} A^m_{\beta}$, only with different non-negative weights.

\section{H. Continuation: Network Initialization and Growth}
\label{sec:h}
The present paper is mainly concerned with {\it accelerating the convergence} of iterative solvers using multilevel methods. The minimization problems under consideration (NN optimization in particular) are highly nonlinear and often have many local minima. The iterations' convergence speed and the solution they approach heavily depend on where (or how) they are {\it initialized}.

A popular initialization approach, and the simplest to implement, is to start {\it random}. Usually such a start is far from a solution and requires many iterations to converge. Moreover, a random start for a highly nonlinear problem, even when by chance successful, would often lead to wrong or unhealthy solution, e.g., the obtained network would not be generalizable enough and would show high sensitivity to slight data perturbations.

Generally, at high nonlinearity, a desired solution can be reliably obtained only if we start from a close-enough first approximation. In many areas of scientific computation, a common way to obtain a good first approximation (and trace sequences of solutions) is by the method of {\it continuation} (also called ``embedding`` or ``Davidenko method''; see for example \cite{keller77}). Namely, a sequence of auxiliary problems is formulated, starting from an easily solvable (e.g., linear) problem and gradually increasing its complexity all the way to the desired problem, such that the (approximate) solution of each problem in the sequence is a good initial guess for the next problem\footnote{reworded this sentence since it was a bit cumbersome; see if you like it.}. The sequence of auxiliary problems in fact {\it defines} which solution we are interested in, {\it even when it is not the lowest minimum} of the formal objective functional. A {\it multilevel} solver integrated with a continuation process may in fact be essential for a good {\it definition} of the desired solution (see example in \cite{SafRon}).

The continuation process need not be expensive, since for dragging the system to the desired attraction basic it may be sufficient so only solve each auxiliary problem very crude, leaving the bulk of computation to the high-accuracy final stage.

The continuation approach is in fact already largely implemented in Neural Network computations, e.g., under the name ``knowledge transfer'' or ``downstream tasks``, initialized by a pre-trained foundation model. But continuation can of course be applied to the pre-training itself.

Choosing for example the objective of pre-training to be the identification when two given images are two different views of the same object or scene, one can start with the easy case that the two images are adjacent video frames, and gradually develop the network to deal with increasingly longer time intervals between the frames \footnote{Simplified sentence, see what you think.}. One can start with only few different objects in the database, perhaps even just one object (like a baby starting by recognizing just his/her mother's face), and gradually increase the number of objects. Similarly, one can gradually increase the number of moving parts of objects, or the noise level, occlusion, distortions, etc. and then introduce object classes and other distinctions of interest.

As the sequence of problems unfolds, the NN architecture can, and should, adapt itself, typically adding more channels, more layers, denser resolutions, wider kernels (see Sec.~\ref{sec:13}), or wider attention context windows, more speciality sub-networks (see Sec.~\ref{sec:15.7}), or stronger localization procedures (see Sec.~\ref{sec:5.4}) -- where and when needed.

A multilevel structure can be particularly useful here, as many continuation steps can be confined to coarse levels, with rare visits to finer ones to update the fine-to-coarse defect correction (see Sec.~\ref{sec:?}). Parameter {\it relations} (like those involved in coarsening) can be more suitable for knowledge transfer than the parameters themselves.

Furthermore, one organic mode of architecture adaptation during continuation can emerge from the multilevel solver: the latter typically introduces {\it fixed} coarsening layers (like $Q$ in Fig.~\ref{fig:2}); then, with more incoming data later in the continuation sequence, these layers can be {\it ``unfrozen''} to become regular {\it learned} layers.

\section{Z}
For a {\it deterministic} {\it linear} system of equations, $Ax = b$, producing low-GER test functions is quite straightforward: just apply your relaxation iterations to the homogeneous equation $Ax = 0$, starting from a random initial approximation $x$. When the iteration slows down the current approximation (which is also the current error $e$, since the solution is $0$) is an example of a {\it relaxed} error, a lower-GER error. Starting with another random error, another low-GER example is produced; and so on. To produce examples with still-lower GER (if needed), apply the to homogeneous equation not just your relaxation, but your current multilevel cycles.

For the deterministic {\it nonlinear} system $A(x) = b$ one can apply the same procedure to the system $A(y) = A(x_0)$, where $x_0$ is the current approximate solution and $y = x_0 + e$, $e$ being initially a random perturbation. The initial as well as all subsequent-iteration residuals $A(x_0) - A(y)$  throughout this procedure should be large compared with $b - A(x_0)$, the current residual, for the procedure to produce a good example of a relaxed error. Also, the iterations should converge $y$ to $x_0$, which is not always guaranteed in nonlinear systems. So the procedure is quite limited in producing high-accuracy examples of relaxed errors. Still, it should be adequate for producing reasonable first approximations to $P$ and $Q$ \footnote{Why?}. And in the stochastic case, good initial coarsening is provided by the sequence of samples described in Sec.~\ref{sec:5.5}. Then, to get more or higher-accuracy examples, one can generally use the following steps.

\section{E1}
(Generally,correlations between variables (or between gradients) converge much sooner than the variables (or gradients) themselves.)

\section{E2}
Similarly, the fine-tuning of a pre-trained model to a new task may be mostly confined to the coarse levels of the pre-trained model.

\subsection{15.6 Memory Efficiency}
The organization of input activations and weights in mini-clusters allow storing only their differences from cluster centers, which can be done in a relatively lower accuracy, resulting in large store savings via quantization and low-dimensional projections (see App.~\ref{G.3}).

\section{F1}
\subsection{6.1 Non-uniform Noise}
In the case that the sampling noise is nearly uniform (i.e., the variance $\sigma^2$ has comparable values for all components), a conventional multigrid-like training algorithm can treat the two problematic aspects of the stochastic ill-conditioning: first, it can apply only a small number of updates at the fine level (using $O(\mu^2/sigma^2)$ minibatch sizes, where $\mu$ is the current target accuracy),

\section{F2}
Often, however, the real problem is the non-uniformity of the noise, where some high-$\lambda$ components may be very noisy (having much larger $\sigma^2$ than others), forcing small $\varepsilon$ in (\ref{eqn:A.7)}. Therefore, the stochastic multilevel algorithm starts with separating low noise from higher noise (see Sec.~\ref{sec:5.6}).

The coarsening with a wider kernel should yield much better coarsening ratios\footnote{Why?}. This, together with running most of the training at coarse levels, and using corresponding upscaling (see Sec.~\ref{sec:upscaling}) at inference time, may compensate for the increased number of parameters in a wider kernel network.

\section{T1}
Bottom-up agglomerative algorithms (see \cite{17} for a comprehensive study) can be used when $n$ is not too large, but for large $n$ their quadratic $O(n^2)$ complexity makes them inefficient. Moreover, bottom-up approaches build on strong {\it binary} correlations, but in may cases no such correlations exist while there still exist larger mini-clusters that can be coarsened (having small CR for small REE). A top-down algorithm, such as RMC described below, should then be used to construct suitable miniclusters. With its nearly linear complexity it can in particular be used to {\it find} the strong binary correlations (in the case that they exist by not apriori known), if needed.

\section{T2}
\subsection{G.1 Recursive Mini Clustering: $S^1 = \text{RMC}(S^0)$}
Given a set $S^0$ of vectors in $\Real^m$, our aim is to construct a set $S^1$ of many disjoint miniclusters whose union is $S^0$. By ``minicluster`` we mean a small subset of highly correlated vectors.

RMC first divides $S^0$ into $k$ (a small number) subsets $(S^0_1,S^0_2,\dots,S^0_k)$ using a standard k-means algorithm \cite{kmeans}, and defines the first approximation to $S^1$ as the union of miniclusters of all subsets. \footnote{Unclear. Miniclusters of $S^0_i$ have not yet been defined here.} This $S^1$ may not be good enough because the k-means division may have placed many close neighbors (highly correlated vectors) in different subsets. An improvement to $S^1$ is obtained by redefining the miniclusters using k-means within each super-minicluster, where the super-miniclusters are obtained by applying RMC to the set $S^1$. (Each member of $S^1$ is not a vector but a mini cluster of vectors, so for RMC it is represented by its center, i.e., the average of its vectors). Indeed, additional improvement steps of this type can be further applied.

{\bf Notation.}
\begin{itemize}
    \item $S^0 = \left\{x_i\right\}_i$: the given set of vectors, $x_i \in \Real^m$.
    \item $S^1 = \left\{x_i^1\right\}_i$: set of miniclusters.
    \item $S^2 = \left\{x_i^2\right\}_i$: set of miniclusters of miniclusters. \footnote{Also denoted super-miniclusters above, which is confusing to a reader.}
    \item $(S^0_1,S^0_2,\dots,S^0_k)$ : subsets of $S^0$.
\end{itemize}

% TODO: replace by algorithmic environment.
{\bf Function $S^1 = \text{RMC}(S^0)$.}
\begin{enumerate}
    \item If $|S^0| < k_1$ or $\text{radius}(S^0) < R_1$, then assign $S^0$ as  a mini-cluster. Otherwise:
    \item $k = \min\left\{k_2, \ceil*{\frac{|S^0|}{k_1}} \right\}\,.$
    \item $\left\{S^0_1,S^0_2,\dots,S^0_k\right\} \longleftarrow \text{k-means}\left(S^0\right)\,.$\footnote{The k-means algorithm here is missing some descriptive details. What is the initial guess for] the centers? How many iterations do we perform?}
    \item $S^1 \longleftarrow \bigcup_{i=1}^k \text{RMC}\left(S^0_i\right)\,.$
    \item $S^2 \longleftarrow \text{RMC}\left(\text{centers}\left(S^1\right)\right)\,.$
    \item $S^1 \longleftarrow \bigcup_{x_j^2 \in S^2} \text{k-means}\left(\bigcup_{x_i^1) \in x^2_j} \text{center}x_i^1 \right)\,.$
    \item Possibly repeat steps 5--6 several times.
    \item Return $S^1$\,.
\end{enumerate}

\section{T5}
\subsection{G.2 Operation Count}
Let $\mu$ be the number of $S^1$ improvement iterations (steps 5--6) applied. Assume for simplicity that $k_1 = k_2 = k$, and that $k$ and $\mu$ are constant at all RMC levels. Let $a k n$ and $b k n$ be the number of operations in the k-means of Steps 3 and 6, respectively. Let $f(n)$ be the operation count of $\text{RMC}(S^0)$. The RMC recursion implies
$$ f(n) = k f\left \frac{n}{k} \right) + a k n + \mu \left(f\left( \frac{n}{k} \right) + b k n\right)\,, \quad f(k^2) = b k ^3\,. $$
Defining $g(n) = f(n)/n$, one gets
\begin{equation}
    g(n) = \left(1 + \frac{\mu}{k}\right) g\left( \frac{n}{k} \right) + (a + \mu b) k\,,
    \quad g(k^2) = b k\,.
    \label{eqn:G.1}
\end{equation}
Assuming $n = k^{\nu}$ and denoting $h_{\nu} = g(n) = g(k^{\nu})$, $r = 1 + \mu/k$ and $A = (a + \mu b)k$, we get
\begin{equation}
    h_{\nu} = r h_{\nu-1} + A\,,\quad h_2 = b k\,,
    \quad g(k^2) = b k\,,
    \label{eqn:G.2}
\end{equation}
which for $\mu > 0$ implies
\begin{equation}
    h_{\nu} = r^{\nu-2} \left( b k + \frac{A}{r-1} \right) - \frac{A}{r-1}\,, \quad \nu \ geq 2.
    \quad g(k^2) = b k\,,
    \label{eqn:G.3}
\end{equation}
Since $\nu = \log n / \log k$, choosing $k$ such that
\begin{equation}
    k \log k \geq \mu \left( \log n - 2 \log k\right),,
    \label{eqn:G.4}
\end{equation}
it follows that $r \leq 1 + \frac{1}{\nu-2}$ and hence $r^{nu-2} \leq e$. Then
\begin{equation}
    g(n) = h_{\nu} \leq e b k + \left( \frac{a}{\mu} + b \right) (e-1) k^2\,.
    \label{eqn:G.5}
\end{equation}
For example, for $\mu = 1$ and $n = 10^3, 10^6, 10^9$ and $10^{12}$, (\ref{G.4}) is satisfied by $k = 4,6,8$ and $10$, respectively. So (\ref{G.5}) yields for $\mu = 1$
\begin{align}
    g(10^3) &\leq& 27.5 a + 38.4 b \\
    g(10^6) &\leq& 62 a + 79 b \\
    g(10^9) &\leq& 110 a + 132 b \\
    g(10^12) &\leq& 172 a + 205 b\,.
    \label{eqn:G.6}
\end{align}
For $\mu = 0$, if we choose $k = 10$, (\ref{G.2}) entails
\begin{equation}
    g(10^{\nu}) \leq 10 (\nu - 2) a + 10 (\nu - 1) b\,.
    \label{eqn:G.7}
\end{equation}
To emphasize the smallness of $a$ and $b$, note that there is no need to fully converge each of the k-means calculations. We are only interested in the smallness of the clusters' average radius, which does not significantly change after just few k-means iterations.

\subsection{G.3 Further Efficiency Notes}
In many (perhaps inmost practical) problems, although the overall dimension $m$ may be high, at each scale of the problem the data is {\it locally low dimensional}. For such problems, even though the local low dimensional space may not explicitly be given, the RMC algorithm should work well even with $\mu = 1$. This is because at low dimension the k-means clusters would have many more points in their interior than close to their boundary, so only few close neighbors will end up belonging to different k-means clusters.

Note that to be successful, the algorithm need not mini-cluster together all pairs of strongly coupled vectors. {\it Particularly} strongly coupled pairs will be miniclustered together with high probability, but many other strongly coupled vectors may not be miniclustered together; they will however belong to strongly coupled miniclusters that will subsequently be clustered together at a higher coarsening level.

In RMC applications that aim at larger miniclusters (more than just a couple of vectors per minicluster) the correction (steps 5--6 of the algorithm) is not likely to significantly change the average cluster radius, so $\mu=0$ can safely be used.

Also, each k-means procedure, especially when larger clusters are sought, can be much accelerated by a {\bf multilevel} approach which first finds cluster centers using only a small random subset instead of the full set of vectors, then improve the center positions using a larger random subset, etc. Only one iteration needs to be performed with the full set of vectors, so the overall work need not be larger than the work of two full k-means iterations.\footnote{Provided that the random subset is doubled at each level of this multilevel procedure.}

Finally, much further acceleration can be gained by {\bf random-projection hashing and quantizations.} Indeed, at the first RMC level, where k-means subdivides the entire set of vectors into few large subsets, low accuracy is sufficient in measuring distances. One can then use a random projection that represents each vector by a much shorter vector, and that short vector can be quantized to a much lower precision. As one proceeds to progressively finer clusters higher accuracy is required, but each vector can then be represented by its difference from the parent-cluster center. Relative to that difference, low accuracy is again sufficient, again allowing representation by shorter and quantized vectors.

Importantly, note that this approach can also yield a {\bf very compact storage}, enabling fitting very long inputs into memory. Each vector is represented by its hashed-and-quantized distance from its minicluster center; tje centers are similarly represented by their hashed-and-quantized differences from the center of the parent cluster; etc.

Note that the RMC algorithm is {\bf highly parallelizable}.

\subsection{G.4 Hidden Miniclusters}
Let again $S^0 = \left\{x_i\right\}_i$ be the set of input activations, $x_i \in \Real^m$. We search for an {\it ``association matrix''} $T \in \Real^{\bar{m} \times m}$ such that the set $\left\{T x_i\right\}_i$ has as good as possible aggregates (miniclusters), by which we mean as small as possible Lagrangian $L_a = E + \lambda n$, where $E$ is a chosen error metric, $n$ is the number of miniclusters, and $\lambda$ is a chosen Lagrange multiplier.

$E$ and $\lambda$ depend on the use case Denoting $u_i = T x_i$, a typical example of $E$ is the ratio
\begin{equation}
    E = \sum_i \left\| u_i - C_{I(i)} \right\| / \sum_i \left\| u_i \right\|
    \label{clustering_error_metric}
\end{equation}
with some chosen norms, where $C_{I(i)}$ is the center of the $I${\it th} minicluster to which $u_i$ belongs. (In calculating derivatives of $E$ one can consider the denominator of (\ref{clustering_error_metric}) and the center $C_I$ to be frozen.) \footnote{Wouldn't it make sense then, in order to define a consistent optimization problem, to define $E$ with a frozen denominator and centers, not just the derivatives of $E$?} The Lagrange multiplier $\lambda$ can be used as a control hyperparameter: choosing lower $\lambda$ would yield smaller \footnote{For smaller $E$, or for the same value of $E$, do we obtain a larger $n$?} $E$ for larger $n$.

Another hyperparameter that depends on the use case is $\bar{m}$, the dimension of the vectors $u_i$. Smaller $\bar{m}$ is desired for compute, storage and data efficiencies, but too small $\bar{m}$ may not give coarsenable aggregates (i.e., small $\text{CR}_{\text{local}}$ for small $\text{REE}_{\text{local}}$; see Sec.~\ref{sec:5.2}).

We this get a combination of two minimization problems, each having its own parameters: minimizing the overall loss functional $L(w)$ and minimizing the Lagrangian $L_a(T)$. We can iteratively solve both of them using the same forward feeding of samples, but with separate backprojections of gradients. We expect $T$ to converge faster and be better initialized by continuation (see App.~\ref{sec:continuation}), since it is less dependent on the specific task, and its backpropagation is only within one layer. Then, as soon as $T$ starts to converge, the revealed hidden aggregates with start helping to produce the right coarse variables for accelerating the convergence of $W$.

\subsection{G.5 Multiple Associations}
Different initializations may yield different association matrices. One should accept all those that yield low values of $L_a$. Each of them would contribute to accelerating the minimization of the loss $L$. Each of them may also lead upon upscaling to the emergence of different high-level concepts. They may be trained and applied simultaneously, in parallel to each other.

A relatively fast multilevel hunt for multiple good (having small $L_a$) association matrices can proceed as follows. Begin by producing many association matrices with small width $\bar{m}$, each starting from a random $T$ followed by relaxation (possibly multilevel-accelerated) with relatively large $\lambda$ (producing partitions into few large miniclusters). Throw out all cases except those that have ended up with a small $L_a$.

For the next level, {\it double} the matrix width $\bar{m}$. Instead of starting from random, each $T$ calculation will start by concatenating the rows of a random pair of association matrices of the previous level. Relax (possibly with multilevel acceleration) now with a reduced value of $\lambda$ (creating finer miniclustering). Throwing out again all the matrices that do not converge to a clustering with small $L_a$, proceed to the next level, again doubling the width $\bar{m}$; and so on. Stop at the first level that yields good coarsenable aggregates (small CR for small REE).

(This is just a preliminary outline of multiple association algorithms. Furthermore, an empirical study is required.)
































%%%%%%%%%%%%%%%%%%%%%%%%%%%%%%%%%%%%%%%%%%%%%%%%%%%%%%%%%%%%%%%%%%%%%%%%%%%%%%%%%%%%%%%
% Bibliography.
%%%%%%%%%%%%%%%%%%%%%%%%%%%%%%%%%%%%%%%%%%%%%%%%%%%%%%%%%%%%%%%%%%%%%%%%%%%%%%%%%%%%%%%

\bibliography{msgd}
\end{document}































